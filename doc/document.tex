%
% UniPress Dokumentation
% $Revision$ 
% $Version$
% $ID$
%
\documentclass[a4paper,10pt,twoside]{article}
\usepackage{ucs}			% for utf8
\usepackage[utf8]{inputenc}	% utf8 input encoding
\usepackage{ngerman}		% new german orthography

%Grafiken
\usepackage[dvips]{graphics}
\usepackage{epsfig}			% for eps grafix
\usepackage{myFig} 			% special grafix handling, see myFig.sty for help

% Quelltext
\usepackage{listings}

%Footnotes without Tabs
\usepackage[bottom,hang]{footmisc}
\setlength{\footnotemargin}{0pt}

%opening
\title{
	UniPress Dokumentation\\
	Version 0.0.1\\
}
\author{
	Christoph Becker \\
	mail@ch-becker.de
}

\begin{document}
\maketitle
\section{Installation}
 Nach dem Entpacken des Quellcodes (mit \texttt{tar
 xzf <datei.tgz>}) kann UniPress bereits im Browser aufgerufen werden.\\
 Es erscheint eine Fehlermeldung (kein Zugriff auf die Datenbank) und die
 Nach\-frage, ob das System bereits installiert ist. Diesem Link folgt man, um auf
 eine Über\-sichts\-seite zu gelangen. Hier werden alle Abhängigkeiten für die
 Installation geprüft. 

 \subsection{Details zur Installationsprüfung}
 Man sollte sich hier von oben Durcharbeiten. Nachfolgend
 ein paar Hinweise zu den gestesteten Daten:
 \begin{description}
	\item[Betriebssystem] dient nur der Information. Bei Windows-Systemen kann es
	zu unbekannten Wechselwirkungen mit dem Rechtemanagement kommen.
%
	\item[PHP-Version] Diese sollte mindesten 4.4.0 betragen, da bei älteren
	Versionen kein einwandfreies Funktionieren gesichert ist. Letztendlich besteht
	auch ein Sicherheitsproblem im Einsatz alter Serversoftware.
%
	\item[Dateisystem] Das Script versucht automatisch den aktuellen lokalen Pfad
	zu ermitteln, dies ist für einige Schreibvorgänge notwendig. Ebenso werden
	verschiedene Verzeichnisse geprüft, ob Schreibrechte vorhanden sind. Das
	\textsf{log/}-Verzeichnis ist für die interne \textsc{Debug}-Funktion notwendig,
	in \textsf{upload/} werden die Artikel abgelegt und nach \textsf{cache/} werden die
	statischen HTML Dateien generiert. Schreibrechte für die Konfigurationsdatei
	\textsf{main.conf.ini} sind für das die Installation abschließende Sichern der
	Konfiguration nötig und können nach der Installation wieder entfernt werden.
    Um Ihnen die Arbeit der Rechtevergabe abzunehmen, existiert ein Script im
    Wurzel\-ver\-zeichnis der installation, \textsf{install.sh}. Rufen Sie es
    einfach auf. 
    %
	\item[Datenbankverbindung] Die Verbindung zur Datenbank ist notwendig um das
	System zu installieren und in Betrieb zu halten. Zunächst wird auf das
	Vorhandensein entsprechender PHP-Module getestet und danach die Zu\-gangs\-da\-ten.
%
	\item[PHP-Modul] Kann hier KEIN Modul gefunden werden, müssen Sie dieses in
	Ihrer \textsf{php.ini}-Datei (meist in \textsf{/etc/php5/apache2/} zu finden)
	ändern. Konsultieren sie hierzu das PHP-Handbuch.
	%
	\item[Verbindung und Datenbankstatus] sind zum Teil abhängig davon, ob sie mit
	einem previvligiertem Datenbank-Account installieren, oder nicht. Editieren Sie
	hierfür mit einem Editor Ihrer Wahl die Datei \textsf{index.php}.
	Details zu dieser Datei finden Sie im nächsten Abschnitt, \textbf{install/index.php}.
 \end{description}
%
%
%
 \subsection{install/index.php}
   In dieser Datei sind alle wichtigen Daten für den weiteren Zugriff zu
   hinterlegen. Ab Zeile 20 geben Sie bitte die Zugangsdaten für die Datenbank
   ein. Existiert noch keine entsprechende Datenbank, Sie besitzen aber einen
   previligierten Datenbankaccount (root), so tragen Sie diese Daten hier ein.\\
   Ab Zeile 25 wird der Datenbankserver (in der Regel \textsf{localhost}) und
   der zu nutzende Datenbankname abgefragt. Geben Sie hier bitte den
   Datenbanknamen ein, der Ihnen von Ihrem Admin zugeteilt wurde oder wählen Sie
   selbst (prev. Account).\\
   Achten Sie darauf, dass \texttt{\$db2['create\_T']} und
   \texttt{\$db2['create\_AU']} jeweils den Wert \texttt{true} annehmen um
   sowohl alle nötigen Tabellen, als auch den Administrator anlegen zu lassen.\\
   An dieser Stelle sind für den unpreviligierten Nutzer alle Möglichkeiten
   erschöpft und er kann die Installationsseite erneut aufrufen. Sind alle
   Voraussetzungen erfüllt, so erscheint am Ende der Seite ein Link zum
   fortsetzen der Installation. Können Sie nicht alle Fehler beheben, denken
   aber, dass dieser Fehler unerheblich ist, so hängen Sie in der Adresszeile
   Ihrers Browsers folgende Argumente an \texttt{?installation=start}.
   
   % 
   %
 \subsection{Extras für previligierte Accounts}
  \begin{description}
 	\item[Nutzer löschen] In Zeile 31 können Sie einen vorhandenen Datenbanknutzer
	löschen lassen, ändern Sie dazu \texttt{false} in \texttt{true} oder übergeben
	Sie beim Scriptaufruf im Browser \texttt{install/index.php?drop\_user=true}.
	%
	\item[unpreviligierten Nutzer erstellen] Um dies zu automatisieren, setzen Sie
	\texttt{\$create} in Zeile 33 auf \texttt{true} oder übergeben Sie 
	\texttt{create\_user=true}
	via GET im Browser. Analog zu Nutzer löschen.
	%
	\item[Zugangsdaten unpreviligierter Benutzer] können ab Zeile 36 hinterlegt
	werden. Wenn dieser Benutzer angelegt werden soll. Dazu muss
	\texttt{create\_user=true} sein.
	
  \end{description}
%
 \subsection{Post-Install}
   Nach der Installation sollten Sie sicherstellen, dass folgende Dinge
   passieren: 
   \begin{itemize}
     \item Löschen des Installationsverzeichnisses oder das entziehen der Rechte
     (chmod 000 install)
     \item Ändern des Besitzers von \textsf{cache} auf den Besitzer des Cronjobs
     und Ändern der Rechte auf 0700.
     \item Abschalten der Debugfunktion nach erfolgreichem Test
     (\texttt{init.php, Zeile 52 auf false setzen})
     \item 
   \end{itemize}
    %
%
\section{Statische Dateien}
Als \textit{Statische Dateien} werden nachfolgend die Listen der Artikel
bezeichnet, die als statische HTML Dateien, also nicht-interaktiv, erzeugt
werden. \\
Um diese Dateien erzeugen zu können, muss ein \textit{Cronjob} eingerichtet
werden, der zu festgelegten Zeiten (1.ooUhr früh) die Datei
\textsf{statiker.php} im Wurzelverzeichnis aufruft. So ein Eintrag wie folgt
aussehen:
%
{\tiny
	\begin{lstlisting}
	# /etc/cron.d/presscache - generates cache files
	# runs every day at 1.00AM
	* 1 * * *     root   /usr/bin/php /var/www/unipress/statiker.php
	\end{lstlisting}
}
%
\subsection{Cache}
Dadurch werden die statischen HTML Dateien in \textsf{cache/} erstellt. Diese 
sind nach folgendem Schema benannt worden. \\
Das Prefix \textsf{\_all\_} steht für \textit{allgemein}. Diese Dateien enthalten
keinen HTML-Kopf beziehungsweise Fuß. Sie können mit Hilfe der PHP-Funktion 
\textsc{include} oder via SSI in ein bestehendes Layout integriert werden.
Dateien, beginnend mit \textsf{\_cus\_} enthalten den vom Admin im Abschnitt 
Bereiche festgelegten Kopf und Fuß. Somit sind sie eigenständige, vollständige 
HTML Dokumente und können direkt aufgerufen werden\footnote{zum Beispiel via 
iFrames}.\\
Um auf diese Dateien zugreifen zu können, ist nur das Kürzel des entsprechenden
Bereiches notwendig. Für ein Bereich (Institut) mit dem Kürzel AB, wird einmal
eine \textsf{\_all\_AB.html} und eine \textsf{\_cus\_AC.html} Datei erstellt.

\subsection{Cache Templates}
In \textsf{t\_cache} befinden sich Templates für die Erstellung individueller
Auflistungen von Presseeinträgen. So ist es möglich, für jeden Bereich ein
anderes Aussehen zu erzielen. Ist keine spezielle Datei für einen Bereich
hinterlegt, so wird immer auf \textsf{\_all\_default.thtml} zurückgegriffen. Die
Syntax ist selbstklärend und kann aus dem Beispiel abgeleitet werden.

\section{Suche}
 Die Suche befindet sich noch im Teststadium und ist somit noch recht
 unausgereift. \\ 
 Der Pfad zur Suchseite lautet \textsf{http://<Installationsadresse>/suche.php}.


\section{Bekannte Probleme}
 \subsection{Benutzer ändern}
  Ein nicht-previligierter Benutzer kann nachträglich nicht in die lokale 
  Datenbank überführt werden.\\
  Das Löschen von Benutzern ist derzeitig ebenfalls nicht möglich. Sollen ihm
  die Rechte entzogen werden, entferne man alle Zuordnungen zu Bereichen.
  %
 \subsection{Admin ist kein Admin mehr}
  Unter bisher ungeklärten Umständen ist es möglich, dass der Administrator, 
  ``admin'',  seine Rechte verliert und nur noch als gewöhnlicher Nutzer 
  agieren kann. Um dies zu beheben muss die Datenbank manuell geändert werden.\\
  An der Konsole ist die mit Hilfe des mysql-Clients im Browser evtl. mit 
  PHP"-My"-Admin möglich. Danach besitzt der Admin wieder alle Rechte.\\
  Beispiel:
%
  {\tiny
	\begin{lstlisting}
  mysql -u<user> -p<pass> [-h<host>] > use <unipress_datenbank>; 
  Database changed 
  > insert into press_admins (id) values (1); 
  Query OK, 1 row affected (0.01 sec) 
  > quit;
	\end{lstlisting}
  }
 \subsection{Passwort vergessen}
  Für Benutzer die in der lokalen Benutzerdatenbank verwaltet werden, kann das 
  Passwort bisher nur direkt in der Datenbank geändert werden.\\
  {\tiny
	\begin{lstlisting}
  mysql -u<user> -p<pass> [-h<host>] > use <unipress_datenbank>; 
  Database changed 
  > update press_user set pass=sha1('mypass') where name='admin'; 
  Query OK, 1 row affected (0.00 sec) Rows matched: 1  Changed: 1  Warnings: 0 
  > quit;
	\end{lstlisting}
  } Dieses SQL-Query ändert das Passwort für den Benutzer ``admin''. Nach der 
  Installation ist das Passwort ``adminpass'' vergeben worden.
%
%
%
\section{Kontakt}
 Vorschläge, Wünsche, Ideen etc. können auf
 \textsf{https://developer.berlios.de/projects/unipress/} hinterlegt oder 
  an cbecker@nachtwach.de mit dem Betreff ``UniPress'' gesendet werden.
 
\end{document}
